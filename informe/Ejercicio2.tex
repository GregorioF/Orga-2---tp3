\section{Ejercicio 2}
\subsection*{Inicialización de la IDT}
\par{En este ejercicio el tema que se va a tratar es el de inicializar la IDT. Para ello, primero se declaró la estructura de la IDT en idt.c, como un arreglo de 256 entradas de idt\_entry.}
\par{En este punto lo que hicimos fue simplemente llenar las primeras 31 entradas de este arreglo, con las interripuciones definidas por Intel. Para ello se utilizó una macro, ya definida, en la cual asignamos a todas las entradas por igual, como segmento de código en el cual se van a ejecutar, el numero 18 dentro de la GDT, por ser este de nivel 0. Y en los atributos, el valor de 0x8E00, lo cual especifica que cada excepción tiene el bit de presencia encendido, nivel de privilegio 0, y que es una interrupción de 32 bits. Y luego a cada entrada dentro del campo offset, le escribimos la dirección del handler de la misma.}
\par{Los handler de las interrupciones fueron declarados en isr.h e implementados en isr.asm. Donde a cada uno, simplemente se le pasó la tarea de imprimir por pantalla el número de interrpución que representa. Para ello en el área de memoria de la pantalla imipimimos el texto que según Intel representa dicha interrupción.}

\clearpage