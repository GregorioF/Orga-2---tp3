\documentclass[hidelinks,a4paper,10pt, nofootinbib]{article}
\usepackage[width=15.5cm, left=3cm, top=2.5cm, right=2cm, left=2cm, height= 24.5cm]{geometry}
\usepackage[spanish]{babel}
\usepackage[utf8]{inputenc}
\usepackage[T1]{fontenc}
\usepackage{xspace}
\usepackage{xargs}
\usepackage{fancyhdr}
\usepackage{lastpage}
\usepackage{caratula}
\usepackage[bottom]{footmisc}
\usepackage{amssymb}
\usepackage{amsmath}
\usepackage{algorithm}
\usepackage{listings}
\usepackage[noend]{algpseudocode}
\usepackage{hyperref} % links en índice
\usepackage{tabularx} % tablas copadas

\usepackage{graphicx}
\usepackage{sidecap}
\usepackage{wrapfig}
\usepackage{caption}
\usepackage{tikz}
\usepackage{alltt}

% facilitates the creation of memory maps. Start address at the bottom, end address at the top.
% syntax: \memsection{end address}{start address}{height in lines}{text in box}
\newcommand{\memsection}[4]{
\bytefieldsetup{bitheight=#3\baselineskip}    % define the height of the memsection
\bitbox[]{10}{
\texttt{#1}     % print end address
\\ \vspace{#3\baselineskip} \vspace{-2\baselineskip} \vspace{-#3pt} % do some spacing
\texttt{#2} % print start address
}
\bitbox{16}{#4} % print box with caption
}


%%fancyhdr
\pagestyle{fancy}
\thispagestyle{fancy}
\addtolength{\headheight}{1pt}
\lhead{Organización del Computador II: TP2}
\rhead{$2º$ cuatrimestre de 2016}
\cfoot{\thepage\ / \pageref{LastPage}}
\renewcommand{\footrulewidth}{0.4pt}
\fancyfoot[LO]{\small{Freidin Gregorio, Taboh Sebastián, Romero Lucía Inés}}
%%caratula
\materia{Organización del Computador II}
\titulo{Trabajo Práctico III}
\subtitulo{System Programming}
\grupo{Grupo: El Arquitecto}

\begin{document}

\thispagestyle{empty}
\materia{Organización del Computador II}
\submateria{Segundo Cuatrimestre de 2016}
\titulo{Trabajo Práctico III}

\integrante{Freidin, Gregorio}{433/15}{gregoriofreidin@gmail.com}
\integrante{Taboh, Sebastián}{185/13}{sebi\_282@hotmail.com}
\integrante{Romero, Lucía Inés}{272/15}{luciainesromero@hotmail.com}

\maketitle
\clearpage

\tableofcontents

\clearpage


\section{GDT}
\par{La Tabla de descriptores global o GDT es la encargada de organizar el sistema de segmentación entre otras cosas. Almacena descriptores los cuales pueden ser de segmento, de TSS, de LDT o de call gate. En nuestro sistema contamos con una estructura, gdt$\_$entry, la cual completamos con descriptores de segmentos y de tss.}
\subsection{Descriptores de Segmento}
\par{
Cada descriptor de segmento nos habilita distintas secciones de la memoria. En el caso del tp, nuestro sistema usa segmentación flat, por ende cada descriptor se maneja con el mismo rango de memoria, pero lo que cambia son los atributos con los que se accede a ella.
\begin{itemize}
\item Descriptor Nulo (requerido por Intel).
\item Descriptores Nulos desde la entrada 1 hasta la 17 (requeridos por el tp).
\item Descriptor de Segmento de Código de nivel 0 (privilegios de supervisor y lectura habilitada).
\item Descriptor de Segmento de Código de nivel 3 (privilegios de usuario y lectura habilitada).
\item Descriptor de Segmento de Datos de nivel 0 (privilegios de supervisor y escritura habilitada).
\item Descriptor de Segmento de Datos de nivel 3 (privilegios de usuario y escritura habilitada).
\item Descriptor de Segmento de video 
\end{itemize}
\medskip
\par{Los descriptores previos tenían como base 0x0000, y como límite, 0x6fff, además, el bit de granularidad seteado, por ende cada uno de los segmentos referencia los primeros 1.75 GB. Excepto por el descriptor de segmento de video, el cual posee como base: 0xb800, límite: 0x0f9f y la granularidad en 0.
Todos tenían el bit de sistema en 1 y el de presente también seteado.
}
\medskip

\subsection{Descriptores de TSS}
Luego tenemos las entradas de las tareas, en las cuales profundizaremos más adelante.
\begin{itemize}
\item Descriptor de la Tarea Inicial.
\item Descriptor de la Tarea Idle.
\item De la entrada 25 hasta la 32, nos encontramos con descriptores de tss para cada una de las tareas de nuestro sistema.
\item De la entrada 33 hasta la 40, nos encontramos con descriptores de tss para cada una de las banderas de nuestro tp.
\end{itemize}


\begin{alltt}
\normalfont
		       Todos estos descriptores mantienen el siguiente formato:
		       
         \textbf{Limite:} 0x67
			   
         \textbf{Base:} Dirección de la tss
                     
         \textbf{Presente:} Seteado
                     
         \textbf{Tipo:} 0x9. Combinado con el bit de sistema en 0, tipo 9 se refiere a un descriptor de TSS.
                
         \textbf{Sistema:} 0x0
                     
         \textbf{DPL:} 0x0. Esto es ya que en nuestro sistema queremos que las tareas sólo puedan ser accedidas por
         el kernel, es decir, que no se pueda "saltar" de una tarea a otra.
			 	
         \textbf{Granularidad:} 0x0
                     
         \textbf{AVL:} 0x0
                     
         \textbf{DB:} 0x1 (32 bits)
                     
         \textbf{L:} 0x0
\end{alltt}
}

\section{Modo Protegido}
\par{Al inicio de nuestro sistema nos encontramos en modo real. Esto es consecuencia de la condición de COMPATIBILIDAD de intel. Al iniciar nuestro sistema tenemos un 8086, nuestro código es de \textbf{16 BITS}; \textbf{NO} hay protección de memoria; podemos utilizar \textbf{TODAS} las instrucciones; AX,CX,DX \textbf{NO} son de propósito general. Y además, sólo podemos direccionar 1 MB de memoria. Por esto, luego de cargar los descriptores de código y datos vamos a pasar a "modo protegido", en el cual contamos con código de 32 bits, protección a memoria, y 4 GB de memoria direccionable.
}
\par{
Para poder realizar esto tenemos que encargarnos de un par de puntos previamente:
\begin{itemize}
\item Cargar el GDTR con la dirección de la GDT (LGDTR)
\item Deshabilitamos las interrupciones \textbf{EXTERNAS} (CLI)
\item Habilitamos A20, es decir, habilitamos el acceso a direcciones de memoria superiores a 2$^{20}$.
\item Seteamos el bit PE del registro CR0. (PE: Protected Mode Enable).
\end{itemize}
Entonces, con el contexto ya armado, ejecutamos la instrucción: \textbf{JMP FAR [selector]:[offset]}. En el caso de nuestro TP, el selector sería 18$<<$3, ya que este es el descriptor de código de nivel 0; y como offset utilizamos una etiqueta la cual se encontraba inmediatamente a continuación de esta instrucción.
}

\clearpage

\section{Paginación}
\par{El sistema de Paginacion de memoria, es una sistema en el cual se organiza la memoria de a paginas con tamaño 4K. Este sistema describe una funcion de mapeo de las direcciones virtuales, obtenidas a travez del sistema de Segmentacion, a direcciones fisicas de memoria. Agregando de paso, nuevos niveles de privilegio, como Supervisor o User, o en que direcciones se puede leer y escribir.}
\par{En este sistema optamos por implementar un sistema de paginacion en dos niveles. Esto significa que, para implementar el mapeo de memoria virtual a fisica se utilizan los siguientes elementos: }
\begin{itemize}
	\item {\bfseries CR3: }
	\par{ En el registro de Control CR3, se va a tener guardado la direccion del Directorio de Paginas que esta actuando actualmente, en un sistema se pueden tener mucho mapeos diferentes, por lo que simplemente para cambiar el esquema de Paginación solo se tiene que remplazar el valor del CR3 por la direccion del Directorio de Paginas deseado.}
	\item {\bfseries Directorio de Páginas: }
	\par{Esta estructura consiste en un arreglo de 1024 entradas, en las cuales cada PD\_entry (entrada del Directorio de paginas) los primeros 20 bits son el prefijo de la direccion en la cual esta ubicada la Tabla de paginas correspondiente  a dicha entrada, como estan alineadas a 4K cada direccion, los ultimos 12 bits de esta direccion se asumen que son 0. Y dentro de cada PD\_entry se utilizan estos bits para establecer atributos, el bit 0 para marcar Presente, el bit 1 para marcar R/W, y el bit 2 para marcar U/S, como los principales atributos mencionados.}
	\item {\bfseries Tabla de Páginas: }
	\par{Estra estructura consiste tambien en un arreglo de 1024 entradas, en la cual cada PT\_entry los primeros 20 bits marcan la posicion donde comienza una pagina de 4K de direcciones fisicas de memoria. Nuevamente como las paginas son de 4K los últimos 12 bits de esta dirección se asumen que son 0, y dentro de la cada entrada de la PT, se utilizan para marcar atributos, los principales son los mismos que los mencionados para la PD}
\end{itemize}

\subsubsection*{Proceso de Traducion Memoria Virtual -> Memoria Fisica}
\par{ Sea D la dirección de la cual quiero obtener la direccion fisica mapeada, el proceso de obtencion es el siguiente.} 
\par {Los primero 10 bits de D (D[31:22]), van a ser utilizados como indice dentro de la Page Directory, en la cual, de la entrada obtenida  se va a sacar la direccion de la Page Table correspondiente a esta direccion.}
\par{ Los segundo 10 bits de D (D[21:12]), van a ser utilizados como indice dentro de la Page Table, en la cual de la entrada obtenida se va a sacar la direccion de la pagina de 4K en donde esta contenida la direccion fisica mapeada a D}
\par{ Por último, los ultimos 12 bits de D (D[11:0]), van a ser utilizados como Offset dentro de la pagina obtenida atravez de la Page Table. Osea la direccion final seria igual a direccion de pagina fisica 4K obtenida de PT + D[11:0]. Todo esto suponiendo que D paso todos los controles implementados en el sistema de paginación.}

\subsubsection*{Paginación De Nuestro Trabajo Práctico}
\par{En primera instancia, inicializamos el mapeo para el codigo en kernel sobre el directorio de paginas ubicado en la direccion 0x27000, donde hacemos identity mapping desde la dirección 0x00000000 a 0x0077FFFF. Asignando en la seccion de atributos de cada entrada el valor de 0x3 que significa Presente = 1, R/W = 1, y nivel Supervisor. Las demas entradas las seteamos en 0.}
\par{Luego el esquema de paginación va a cambair según la tarea que se este ejectuando, para ello inicializamos varios directorios de paginas, en el cual generamos un mapping segun corresponda a la tarea que corra actualmente, donde mantenemos el mapeo hecho para el directorio del Kernel, pero agregando paginas nivel Usuario donde la tarea se va a ejecutar.}




\newpage
\section{IDT}
\label{subsec:IDT}
\par{La IDT, Interruption Descriptor Table, se encarga de almacenar descriptores de rutinas de atención sobre interrpuciones de software, de hardware, y expeciones. Este se representa como un arreglo de maximo 2**13 entradas, de tamaño 64 bits. Estos Descriptores cargan informacion sobre, donde se hubica dicha rutina de atencion, atributos de presente, o DPL (nivel de privilegio requerido para acceder a dicha interrupcion), y el selector de segmento que se va a utilizar, en general este selector es el de Código nivel 0, ya que una interrpucion suele necesitar de accesos a nivel Kernel.}

\subsection{Rellenado de IDT}
\begin{itemize}
	\item {\bfseries Expeciones: }
	\par{Por restricciones de Intel, desde la posicion 0 hasta la 19 es donde se van a hubicar los descriptores de rutinas de atencion sobre expeciones, segun el orden que indica el manual de Intel. Estas tienen los atributos de presente en 1 y el DPL en 3, pues no queremos que ninguna tarea de nivel usuario llame a una Exepcion intencionalmente. Luego las entradas entre la 20 y la 31 estan reservadas a futuras posible exepciones, o usos que Intel les proporcione.}
	
	\item {\bfseries Interrupciones de Hardware: }
	\par{ Estas se ubican en las primeras entradas libres a partir de la 32, en particular la 32 es la interrupcion de Reloj, y la 33 la interrupcion de teclado. Ambas entradas en nuestro TP estan seteadas con el SegSel = Cod\_L0, el bit de presente en 1, y DPL = 0.}
	
	\item {\bfseries Interrpuciones de Software: }
	\par{Estas son seteadas a partir de posiciones mayores, en nuestro caso implementamos 2, en la entrada 80 y en la entrada 102. Donde la unica diferencia con las de Hardware respecto a los atributos, es que estas contienen DPL = 3, ya que queremos que puedan ser llamadas por tareas nivel usuario para ejecutar un codigo en nivel 0, esto en particular se le da el nombre de syscall.}

\end{itemize}

\subsection{Proceso de identificación de interrupción requerida}
\par{Cuando ocurre una interrpcion / exepción, el proceso basicamente es obtener el descriptor de interrupcion señalado por el Vector de Interrpciones, y de ahi saltar directamente a la rutina de atención. Ahora la forma en la que se carga dicho Vector es, para expeciones, el mismo procesador se encarga de indentificar cual es la posicion asignada a dicha interrupcion, dentro de la IDT segun Intel, y carga el vector con dicho valor.  Para interrupciones de Hardware se utiliza lo que es el PIC de interrupciones donde cuando un elemento externo solicita una interrpucion, estos estan ordenados en un orden especifico segun un handler de hardware, y este identifica cual de los aparatos fue el que la solicito, cargando de esta manera al Vector de Interrupciones. Y por último a una interrpucion de Software simplemente es el numero q le sigue a la instruccion "int" que es la utilizada para solicitar una interrpucion especifica.}

\subsection{Excepciones}
\par{ESTA ME DA IGUAL}
\subsection{Interrupción de reloj}
\subsection{Interrupción de teclado}
\par{ESTA LA HACE LUCIA}
\subsection{Int 80}
\par{ESTA LA HACE LUCIA}
\subsection{Int 102}
\par{ESTA LA HACE LUCIA}

\section{TSS}
\newpage
\section{Scheduler}
\par{El scheduler es quien se encarga de determinar qué tarea debe ser ejecutada. Para esto contamos con dos arreglos de enteros de 8 posiciones, en el caso de nuestro TP, en el cual se almacena qué navíos/banderas pueden ejecutarse. Cada posición i representa al navío/bandera i+1, y si en dicha posición hay un 1 es que el navío/bandera continúa presente en el scheduler, y, si hay un 0, no. En el caso de nuestro TP algún navío o bandera puede cometer algún error, ya sea una excepción, en cuyo caso como vimos en la sección \ref{subsec: IDT} el handler de la interrupción se encargará de ello; o, en el caso de las banderas, estas pueden haber sido interrumpidas por el reloj antes de pasar por la int 0x50. Cualquiera sea el caso, esto conlleva a la eliminación del navío y bandera correspondiente, por esto es que contamos con los arreglos que nos determinan las posibles "siguiente tarea".}
\par{
Como bien vimos en la sección \ref{subsec: IDT}, cuando debemos eliminar un navío y su bandera, lo único que hacemos es eliminar el índice correspondiente en el scheduler con la función \textbf{inhabilitar$\_$tarea(uint error, int n)}, la cual además, nos imprime en la pantalla de estados el error que se ha cometido.
}
\par{Como el scheduler se comprende de 2 arreglos, contamos con 2 índices, una para cada uno de ellos; \textbf{current} de navíos, y \textbf{currentBanderas}. Contamos con dos funciones las cuales nos proveen el siguiente navío o bandera a ejecutarse, \textbf{sched$\_$proximo$\_$indice} y \textbf{sched$\_$proxima$\_$bandera}. A continuación mostramos un pseudocódigo que aplica a ambas funciones pero con sus respectivos \textbf{currents} y con la diferencia que \textbf{sched$\_$proxima$\_$bandera} se reinicia al llegar a 8.}

\begin{algorithm}[h!]
\caption{sched$\_$proximo$\_$indice}
\begin{algorithmic}
	\State current += 1
	\State i = 0
	\While{ tareas[current]$\%$8 == 0 and i$<$9 }
		\State current += 1
		\State i += 1
	\EndWhile
	\If{tareas[current] == 0}
	\State current = -1
	\EndIf
	\State \textbf{return} current
\end{algorithmic}
\end{algorithm}

\par{Como vimos en la sección \ref{subsec:IDT}, en nuestro TP la interrupción externa 32, la interrupción de reloj, funciona como regulador de los tiempos de las tareas. Cada navío tiene un tick de reloj para correr, si llama a una syscall, el tiempo restante de su quantum es destinado a la tarea idle. Las banderas, por su parte, también tienen un tick de reloj para ejecutarse, con la salvedad de que antes de cumplido este tiempo tienen que llamar a la syscall 0x50. En cada interrupción de reloj se determina cuál será la próxima tarea a ejecutarse.}
\par{En particular, en nuestro TP, tenemos determinado que cada 3 quantum de tareas se corren \textbf{TODAS} las banderas "disponibles", es decir, las que no han sido eliminadas del scheduler por algún tipo de error.}
\par{En resumen, las tareas/navíos se ejecutan en orden, cada 3 quantum se ejecutan las banderas que siguen en el scheduler, al cometer algún error el navío/bandera es eliminado del scheduler junto con su bandera/navío, la función proximo$\_$indice nos devuelve el índice de la tarea a ejecutarse, si es que la hay.} 
\medskip
\par{\textbf{¿Y qué sucede cuando no quedan tareas a ejecutarse?}}
\par{Sigue corriendo indefinidamente la tarea idle. Situación que efectivamente ocurre en nuestro TP.}

\newpage

\section{Ejercicio 1}
\par{El primer ejercicio consistió en dos elementos, el primero fue sobre llenar la Global Descriptor Table (GDT), con ciertos segmentos y el segundo, en pasar a modo protegido.}

\subsection*{Inicializar la GDT}
\par{En esta parte lo que hicimos primero fue crear un arreglo de gdt\_entry, en el archivo gdt.c de 41 entradas y luego completamos 6 de las mismas. En la primera posición del arreglo fue seteado el descriptor Nulo por restricciones de Intel. Y los otros 5 descriptores de segmento fueron completados empezando desde la posición 18, por restricciones del TP,  hasta la 22 inclusive.
\par{Los dos descriptores de segmento de las posiciones 18 y 19 fueron seteados como segmentos de código nivel 0 y 3 respectivamente. Empezando la base de ambos desde la posición 0x0000, y con un límite de 0x6ffff con el bit de Granularity activado respresentando así 1.75 GB.}
\par{\textcolor{red}{1.75 GB son $\frac{7}{4}*2^{30} B = 7*2^{28} B$. Como hay 20 bits del campo ``Límite'' no iban a ser suficientes para direccionar 1.75 GB como se buscaba, entonces por eso se puso un 1 en el bit de Granularidad para que el Límite se refiriera a bloques de 4 KB. Así, se obtiene $\dfrac{7*2^{28} B}{4 KB} = 7*2^{16}$ y restándole 1 se consigue 0x6FFFF, el contenido que se puso en el campo ``Límite''.}}
\par{Al type de ambos segmentos se les puso el valor de 0xA (Exectute/Read), y los atributos  de Sistema y Presencia en 1.}
\par{Los dos descriptores de segmento de las posiciones 20 y 21 fueron seteados como segmentos de datos nivel 0 y 3 respectivamente. Empezando la base de ambos desde la posición 0x0000, y con un límite de 0x6ffff con el bit de Granularity activado. Al type de ambos segmentos se les puso el valor de 0x02 (Read/Write), y los atributos de Sistema y Presencia en 1.}
\par{Por último el descriptor seteado en la posición 22 del arreglo, fue colocado como un descriptor de video, con la base a partir de 0xb8000, y límite 0x0f9f, con el bit de Granularity en 0. A este segmento también se le asignó en los atributos de Sistema y Presencia el valor 1.}


\subsection*{Pasar a Modo Protegido}
\par{Luego de cargar la GDT, creándola con los requisitos ya especificados y cargando el GDT register con \textbf{lgdt}, habilitamos la línea A20 para poder acceder a las posiciones mayores a $2^{20}$ y seteamos el bit de PE de CR0 en 1.}
\par{Así ya con todo el contexto armado, ejectuamos la instrucción \textbf{jmp} 18 $<<$ 3:mp para hacer un jmp far a modo protegido donde ponemos dentro del selector de segmento de código el valor 18 que es la posición dentro de nuestro arreglo gdt, donde está el desciptor de código nivel 0.}

\clearpage

\section{Ejercicio 2}
\subsection*{Inicialización de la IDT}
\par{En este ejercicio el tema que se va a tratar es el de inicializar la IDT. Para ello, primero se declaró la estructura de la IDT en idt.c, como un arreglo de 256 entradas de idt\_entry.}
\par{En este punto lo que hicimos fue simplemente llenar las primeras 31 entradas de este arreglo, con las interripuciones definidas por Intel. Para ello se utilizó una macro, ya definida, en la cual asignamos a todas las entradas por igual, como segmento de código en el cual se van a ejecutar, el numero 18 dentro de la GDT, por ser este de nivel 0. Y en los atributos, el valor de 0x8E00, lo cual especifica que cada excepción tiene el bit de presencia encendido, nivel de privilegio 0, y que es una interrupción de 32 bits. Y luego a cada entrada dentro del campo offset, le escribimos la dirección del handler de la misma.}
\par{Los handler de las interrupciones fueron declarados en isr.h e implementados en isr.asm. Donde a cada uno, simplemente se le pasó la tarea de imprimir por pantalla el número de interrpución que representa. Para ello en el área de memoria de la pantalla imipimimos el texto que según Intel representa dicha interrupción.}

\clearpage

\section{Ejercicio 3}
\par{}

\subsection*{Buffer de video}
\par{}


\subsection*{Inicialización del directorio y de la tabla de páginas para el \textit{kernel}}
\par{}

\subsection*{Activación de paginación}
\par{}


\clearpage

\section{Ejercicio 4}
\par{}

\subsection*{Inicialización del directorio y de la tabla de páginas para tareas}
\par{}


\subsection*{Mapeo y desmapeo de páginas de memoria}
\par{}

\subsection*{Construcción del mapa de memoria para tareas}
\par{}

\clearpage

\section{Ejercicio 5}
\par{}

\subsection*{Entradas en la IDT}
\par{Por el posible caso de que se modificaran los registros dentro de la interrupción, será necesario salvar el estado de los registros al entrar en la rutina (con la instrucción \textbf{pushad}) y restaurarlos antes de salir de ella (con
la instrucción \textbf{popad}).}


\subsection*{Rutina asociada a la interrupción de reloj}
\par{Después de llamar a la rutina \textbf{proximo\_reloj}, que se encarga de mostrar cada vez que se llame, la animación de un cursor rotando en la esquina inferior derecha de la pantalla, y antes del fin de la rutina, debemos rehabilitar las interrupciones, y esto lo hacemos con la instrucción \textbf{call fin\_intr\_pic1}.}
\begin{lstlisting}[language={[x86masm]Assembler}]
global _isr32
_isr32:
    pushad
    call proximo_reloj
    call fin_intr_pic1
    popad
    iret
\end{lstlisting}

\subsection*{Rutina asociada a la interrupción de teclado}
\par{Esta rutina primero lee utilizando la instrucción \textbf{in al, 0x60} el valor almacenado en el puerto 0x60, que dependerá de si la acción realizada en el teclado sea apretar una tecla o soltarla y de qué tecla sea.}
\par{Este valor queda en el registro \textbf{eax}, que se pushea para luego llamar a \textbf{print\_numerito}, una función que realiza lo que corresponda sobre la pantalla.}
\par{Finalmente, se informa al PIC que se pueden rehabilitar las interrupciones.}
\begin{lstlisting}[language={[x86masm]Assembler}]
global _isr33
_isr33:
    pushad
    xor eax, eax
    in al, 0x60
    push eax
    call print_numerito
    add esp, 4
    call fin_intr_pic1
    popad
    iret
\end{lstlisting}

\subsection*{Rutinas asociadas a la interrupciones 0x50 y 0x66}
\par{}


\clearpage

\section{Ejercicio 6}
\par{}

\subsection*{Definición de las entradas en la GDT para las tareas}
\par{}


\subsection*{}
\par{}

\clearpage

\section{Ejercicio 7}
\par{}

\subsection*{Inicialización de las estructuras de datos del \textit{scheduler}}
\par{}


\subsection*{Función sched\_proximo\_indice()}
\par{Primero declaramos dos arreglos de 8 posiciones, uno de tareas y otro de banderas, para indicar si las tareas estaban vivas. Además, usamos un contador current para saber qué tarea correspondía correr.}
\begin{lstlisting}[language={C}]
int tareas[8]={1,1,1,1,1,1,1,1};
int banderas[8]={1,1,1,1,1,1,1,1};
int current = -1;
int currentBanderas = -1;
\end{lstlisting}

\par{Luego implementamos la función \textbf{sched\_proximo\_indice()}. Esta suma 1 a current y entra en un ciclo en el que se aumenta 1 mientras se encuentren tareas que no estén vivas. Si se recorren todas y no hay ninguna para correr entonces current vuelve a -1, en otro caso queda en current mod 8.}
\begin{lstlisting}[language={C}]
short sched_proximo_indice() {
	current +=1 ;
	int i = 0;
	while(tareas[current%8] == 0 && i < 20){
		current +=1;
		i = i+1;
	}
	if(i == 20) return -1;
	return current %8;
}
\end{lstlisting}

\subsection*{Función sched\_proxima\_bandera()}
\par{Usamos un contador currentBanderas para saber qué bandera correspondía correr.}
\begin{lstlisting}[language={C}]
short sched_proxima_bandera(){
	currentBanderas +=1;

	while(tareas[currentBanderas] == 0 ){
		currentBanderas +=1;
		if(currentBanderas == 8) {
			currentBanderas = -1;
			break;
		}
	}
	if( 7 < currentBanderas ) currentBanderas = -1;
	return currentBanderas;
}
\end{lstlisting}

\clearpage

\end{document}
