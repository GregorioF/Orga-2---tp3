\documentclass[hidelinks,a4paper,10pt, nofootinbib]{article}
\usepackage[width=15.5cm, left=3cm, top=2.5cm, right=2cm, left=2cm, height= 24.5cm]{geometry}
\usepackage[spanish]{babel}
\usepackage[utf8]{inputenc}
\usepackage[T1]{fontenc}
\usepackage{xspace}
\usepackage{xargs}
\usepackage{fancyhdr}
\usepackage{lastpage}
\usepackage{caratula}
\usepackage[bottom]{footmisc}
\usepackage{amssymb}
\usepackage{amsmath}
\usepackage{algorithm}
\usepackage{listings}
\usepackage[noend]{algpseudocode}
\usepackage{hyperref} % links en índice
\usepackage{tabularx} % tablas copadas

\usepackage{graphicx}
\usepackage{sidecap}
\usepackage{wrapfig}
\usepackage{caption}
\usepackage{tikz}

% facilitates the creation of memory maps. Start address at the bottom, end address at the top.
% syntax: \memsection{end address}{start address}{height in lines}{text in box}
\newcommand{\memsection}[4]{
\bytefieldsetup{bitheight=#3\baselineskip}    % define the height of the memsection
\bitbox[]{10}{
\texttt{#1}     % print end address
\\ \vspace{#3\baselineskip} \vspace{-2\baselineskip} \vspace{-#3pt} % do some spacing
\texttt{#2} % print start address
}
\bitbox{16}{#4} % print box with caption
}


%%fancyhdr
\pagestyle{fancy}
\thispagestyle{fancy}
\addtolength{\headheight}{1pt}
\lhead{Organización del Computador II: TP2}
\rhead{$2º$ cuatrimestre de 2016}
\cfoot{\thepage\ / \pageref{LastPage}}
\renewcommand{\footrulewidth}{0.4pt}
\fancyfoot[LO]{\small{Freidin Gregorio, Taboh Sebastián, Romero Lucía Inés}}
%%caratula
\materia{Organización del Computador II}
\titulo{Trabajo Práctico III}
\subtitulo{System Programming}
\grupo{Grupo: El Arquitecto}

\begin{document}

\thispagestyle{empty}
\materia{Organización del Computador II}
\submateria{Segundo Cuatrimestre de 2016}
\titulo{Trabajo Práctico III}

\integrante{Freidin, Gregorio}{433/15}{gregoriofreidin@gmail.com}
\integrante{Taboh, Sebastián}{185/13}{sebi\_282@hotmail.com}
\integrante{Romero, Lucía Inés}{272/15}{luciainesromero@hotmail.com}

\maketitle
\clearpage

\tableofcontents

\clearpage
\section{Ejercicio 1}
\par{El primer ejercicio consistió en dos elementos, el primero fue sobre llenar la Global Descriptor Table (GDT), con ciertos segmentos y el segundo, en pasar a modo protegido.}

\subsection*{Inicializar la GDT}
\par{En esta parte lo que hicimos primero fue crear un arreglo de gdt\_entry, en el archivo gdt.c de 41 entradas y luego completamos 6 de las mismas. En la primera posición del arreglo fue seteado el descriptor Nulo por restricciones de Intel. Y los otros 5 descriptores de segmento fueron completados empezando desde la posición 18, por restricciones del TP,  hasta la 22 inclusive.
\par{Los dos descriptores de segmento de las posiciones 18 y 19 fueron seteados como segmentos de código nivel 0 y 3 respectivamente. Empezando la base de ambos desde la posición 0x0000, y con un límite de 0x6ffff con el bit de Granularity activado respresentando así 1.75 GB.}
\par{\textcolor{red}{1.75 GB son $\frac{7}{4}*2^{30} B = 7*2^{28} B$. Como hay 20 bits del campo ``Límite'' no iban a ser suficientes para direccionar 1.75 GB como se buscaba, entonces por eso se puso un 1 en el bit de Granularidad para que el Límite se refiriera a bloques de 4 KB. Así, se obtiene $\dfrac{7*2^{28} B}{4 KB} = 7*2^{16}$ y restándole 1 se consigue 0x6FFFF, el contenido que se puso en el campo ``Límite''.}}
\par{Al type de ambos segmentos se les puso el valor de 0xA (Exectute/Read), y los atributos  de Sistema y Presencia en 1.}
\par{Los dos descriptores de segmento de las posiciones 20 y 21 fueron seteados como segmentos de datos nivel 0 y 3 respectivamente. Empezando la base de ambos desde la posición 0x0000, y con un límite de 0x6ffff con el bit de Granularity activado. Al type de ambos segmentos se les puso el valor de 0x02 (Read/Write), y los atributos de Sistema y Presencia en 1.}
\par{Por último el descriptor seteado en la posición 22 del arreglo, fue colocado como un descriptor de video, con la base a partir de 0xb8000, y límite 0x0f9f, con el bit de Granularity en 0. A este segmento también se le asignó en los atributos de Sistema y Presencia el valor 1.}


\subsection*{Pasar a Modo Protegido}
\par{Luego de cargar la GDT, creándola con los requisitos ya especificados y cargando el GDT register con \textbf{lgdt}, habilitamos la línea A20 para poder acceder a las posiciones mayores a $2^{20}$ y seteamos el bit de PE de CR0 en 1.}
\par{Así ya con todo el contexto armado, ejectuamos la instrucción \textbf{jmp} 18 $<<$ 3:mp para hacer un jmp far a modo protegido donde ponemos dentro del selector de segmento de código el valor 18 que es la posición dentro de nuestro arreglo gdt, donde está el desciptor de código nivel 0.}

\clearpage

\section{Ejercicio 2}
\subsection*{Inicialización de la IDT}
\par{En este ejercicio el tema que se va a tratar es el de inicializar la IDT. Para ello, primero se declaró la estructura de la IDT en idt.c, como un arreglo de 256 entradas de idt\_entry.}
\par{En este punto lo que hicimos fue simplemente llenar las primeras 31 entradas de este arreglo, con las interripuciones definidas por Intel. Para ello se utilizó una macro, ya definida, en la cual asignamos a todas las entradas por igual, como segmento de código en el cual se van a ejecutar, el numero 18 dentro de la GDT, por ser este de nivel 0. Y en los atributos, el valor de 0x8E00, lo cual especifica que cada excepción tiene el bit de presencia encendido, nivel de privilegio 0, y que es una interrupción de 32 bits. Y luego a cada entrada dentro del campo offset, le escribimos la dirección del handler de la misma.}
\par{Los handler de las interrupciones fueron declarados en isr.h e implementados en isr.asm. Donde a cada uno, simplemente se le pasó la tarea de imprimir por pantalla el número de interrpución que representa. Para ello en el área de memoria de la pantalla imipimimos el texto que según Intel representa dicha interrupción.}

\clearpage

\section{Ejercicio 3}
\par{}

\subsection*{Buffer de video}
\par{}


\subsection*{Inicialización del directorio y de la tabla de páginas para el \textit{kernel}}
\par{}

\subsection*{Activación de paginación}
\par{}


\clearpage

\section{Ejercicio 4}
\par{}

\subsection*{Inicialización del directorio y de la tabla de páginas para tareas}
\par{}


\subsection*{Mapeo y desmapeo de páginas de memoria}
\par{}

\subsection*{Construcción del mapa de memoria para tareas}
\par{}

\clearpage

\section{Ejercicio 5}
\par{}

\subsection*{Entradas en la IDT}
\par{Por el posible caso de que se modificaran los registros dentro de la interrupción, será necesario salvar el estado de los registros al entrar en la rutina (con la instrucción \textbf{pushad}) y restaurarlos antes de salir de ella (con
la instrucción \textbf{popad}).}


\subsection*{Rutina asociada a la interrupción de reloj}
\par{Después de llamar a la rutina \textbf{proximo\_reloj}, que se encarga de mostrar cada vez que se llame, la animación de un cursor rotando en la esquina inferior derecha de la pantalla, y antes del fin de la rutina, debemos rehabilitar las interrupciones, y esto lo hacemos con la instrucción \textbf{call fin\_intr\_pic1}.}
\begin{lstlisting}[language={[x86masm]Assembler}]
global _isr32
_isr32:
    pushad
    call proximo_reloj
    call fin_intr_pic1
    popad
    iret
\end{lstlisting}

\subsection*{Rutina asociada a la interrupción de teclado}
\par{Esta rutina primero lee utilizando la instrucción \textbf{in al, 0x60} el valor almacenado en el puerto 0x60, que dependerá de si la acción realizada en el teclado sea apretar una tecla o soltarla y de qué tecla sea.}
\par{Este valor queda en el registro \textbf{eax}, que se pushea para luego llamar a \textbf{print\_numerito}, una función que realiza lo que corresponda sobre la pantalla.}
\par{Finalmente, se informa al PIC que se pueden rehabilitar las interrupciones.}
\begin{lstlisting}[language={[x86masm]Assembler}]
global _isr33
_isr33:
    pushad
    xor eax, eax
    in al, 0x60
    push eax
    call print_numerito
    add esp, 4
    call fin_intr_pic1
    popad
    iret
\end{lstlisting}

\subsection*{Rutinas asociadas a la interrupciones 0x50 y 0x66}
\par{}


\clearpage

\section{Ejercicio 6}
\par{}

\subsection*{Definición de las entradas en la GDT para las tareas}
\par{}


\subsection*{}
\par{}

\clearpage

\section{Ejercicio 7}
\par{}

\subsection*{Inicialización de las estructuras de datos del \textit{scheduler}}
\par{}


\subsection*{Función sched\_proximo\_indice()}
\par{Primero declaramos dos arreglos de 8 posiciones, uno de tareas y otro de banderas, para indicar si las tareas estaban vivas. Además, usamos un contador current para saber qué tarea correspondía correr.}
\begin{lstlisting}[language={C}]
int tareas[8]={1,1,1,1,1,1,1,1};
int banderas[8]={1,1,1,1,1,1,1,1};
int current = -1;
int currentBanderas = -1;
\end{lstlisting}

\par{Luego implementamos la función \textbf{sched\_proximo\_indice()}. Esta suma 1 a current y entra en un ciclo en el que se aumenta 1 mientras se encuentren tareas que no estén vivas. Si se recorren todas y no hay ninguna para correr entonces current vuelve a -1, en otro caso queda en current mod 8.}
\begin{lstlisting}[language={C}]
short sched_proximo_indice() {
	current +=1 ;
	int i = 0;
	while(tareas[current%8] == 0 && i < 20){
		current +=1;
		i = i+1;
	}
	if(i == 20) return -1;
	return current %8;
}
\end{lstlisting}

\subsection*{Función sched\_proxima\_bandera()}
\par{Usamos un contador currentBanderas para saber qué bandera correspondía correr.}
\begin{lstlisting}[language={C}]
short sched_proxima_bandera(){
	currentBanderas +=1;

	while(tareas[currentBanderas] == 0 ){
		currentBanderas +=1;
		if(currentBanderas == 8) {
			currentBanderas = -1;
			break;
		}
	}
	if( 7 < currentBanderas ) currentBanderas = -1;
	return currentBanderas;
}
\end{lstlisting}

\clearpage

\end{document}
