\section{Ejercicio 6}
\par{A cada tarea le corresponden dos TSS, una para el código de la tarea y otra para la ejecución de la función “bandera”.}

\subsection*{Definición de las entradas en la GDT para las tareas}
\par{Cada tarea creada necesita tener su descriptor de TSS en la GDT. Este nos permite acceder a la TSS de cada tarea, que definimos completando los campos del struct tss que se declara en tss.h. Guardamos la tarea\_inicial en la posición 23 de la GDT y la tarea Idle en la posición 24 (valores referenciados respectivamente por las constantes globales TAREA\_INICIAL y TAREA\_IDLE definidas en defines.h). Las nuevas entradas de la GDT son inicializadas al llamar desde kernel.asm a la función tss\_inicializar implementada en tss.c.}

\par{Se completaron los campos de los descriptores de las TSS de la siguiente forma:}
\begin{itemize}
\item[•] \textbf{Límite:} Como el tamaño de la estructura tss es 104, al restarle 1 obtenemos 0x67.
\item[•] \textbf{Tipo:} Este campo tiene 4 bits: 10B1, donde B (\textit{busy}) representa que esté en uso la tarea asociada a este descriptor. Inicialmente la tarea no estará cargada así que B se seteará en 0, por lo que Tipo contendrá 0x9.
\item[•] \textcolor{red}{\textbf{Nivel de privilegio:} Queremos que sólo el \textit{kernel} pueda realizar el cambio de tarea a esta, por lo que completamos con 0x0.}
\item[•] \textbf{Presente:} Este campo tiene 0x1 dado que hay una tss asociada a este descriptor.
\item[•] \textbf{System:} En este campo se pone 0 cuando se trata de contenidos de sistema como ser gdt, ldt, tss y 1 cuando son datos o código, por lo que se puso 0x0.
\item[•] \textbf{db:} Como se trabaja en 32 bits, se cargó 0x1.
\end{itemize}
\par{Los demás campos se cargaron en 0.}


\subsection*{TSS de Idle}
\par{Como la tarea Idle se encuentra en la dirección 0x00020000, pusimos en eip 0x20000. Además, como la pila se aloja en la página 0x0002A000 y será mapeada con identity mapping, completamos ebp y esp con 0x2A000.}

\subsection*{TSS de las demás tareas}
\par{Como el código de las tareas está mapeado a partir de la dirección 0x40000000 completamos el eip de los navíos con esa dirección.}

\subsection*{Código para la ejecución de la tarea Idle}
\par{}


\clearpage