\section{TSS}
\par{El Task State Segment o TSS, es una estructura especial del sistema utilizada para guardar los contextos de ejecución de una tarea al momento de dejarla para saltar a otra. Y ésta justamente comenzará con el contexto guardado en la TSS respectiva.}
\par{ El tamaño de la TSS es de 104 Bytes, donde se guardan todos los registros de 32 bits, todos los selectores de segemento y el puntero al comienzo de la pila de Nivel 0, 1, 2 con sus respectivos SegSel, que seran utilizados en caso de llamar a una interrupcion de diferente nivel a la tarea.}

\subsection{Proceso de Inicialización de Tareas}
\par{En primer lugar el sistema de multi-tasking requiere que haya una TSS destinada para la tarea inicial; el propósito de ésta TSS es el de guardar el contexto previo a saltar a la primer tarea a ejecutar.}
\par{Luego como mencionamos anteriormente, cada tarea comienza a ejecutar según el contexto de su TSS asignada, por lo que no puede haber valores \textsl{fruta} para ciertos registros escenciales como CR3, EIP, CS, etc... Por ello dentro de la sección de código de kernel llamamos a la función \textbf{tss\_inicializar} en la cual recorremos todas las TSS y las rellenamos con los valores correspondientes a los elementos escenciales de la siguiente manera:  }

\begin{itemize}
	\item \textbf{TSS\_Tareas}
	\begin{alltt}
		\item  \textbf{EIP} = 0x40000000  - por enunciado
		\item  \textbf{ESP} = 0X40001C00  - por enunciado
		\item  \textbf{EBP} = 0X40001C00  - por enunciado
		\item  \textbf{EFLAGS} = 0X202  - permite interrupciones
		
		\item  \textbf{DS} = DAT_L3  - es tarea de nivel 3
		\item  \textbf{CS} = COD_L3  - es tarea de nivel 3
		
		\item  \textbf{CR3} = PD correspondiente
		
	\end{alltt}
	
	\item \textbf{Tss\_Banderas}
	\begin{alltt}
		\item \textbf{EIP} = Posicion correspondiente dentro de la funcion de la Tarea a la que corresponde
		\item \textbf{ESP} = 0X40001FFC  - por enunciado
		\item \textbf{EBP} = 0X40001FFC  - por enunciado
		\item \textbf{EFLAGS} = 0X202  - permite interrpuciones
		
		\item \textbf{DS} = DAT_L3  - es tarea de nivel 3
		\item \textbf{CS} = COD_L3  - es tarea de nivel 3
		
		\item \textbf{CR3} = PD de la tarrea a la cual corresponde 
	\end{alltt}
\end{itemize}


\par{Por último a cada Page Directory de cada tarea, los incializamos segun como pide el enunciado, haciendo identity mapping como en el kernel y agregando el mapeo a partir de la posicion 0x40000000, a las páginas físicas a partir de la dirección 0x100000 y a éstas copiamos el código de las tareas el cual se encuentra a partir de la dirección física 0x10000.}