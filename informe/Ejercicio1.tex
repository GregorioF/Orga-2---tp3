\section{Ejercicio 1}
\par{El primer ejercicio consistió en dos elementos, el primero fue sobre llenar la Global Descriptor Table (GDT), con siertos segmentos y el segundo, en pasar a modo protegido.}

\subsection{Inicializar la GDT}
\par{En esta parte lo que hicimos primero fue crear un arreglo de $gdt_entry$, en el archivo gdt.c de 41 entradas y luego completamos 6 de las mismas. La primer posicion del arreglo fue seteado el descriptor Nulo por restricciones de Intel. Y los otros 5 descriptores de segmento fueron completados empezano desde la posicion 18, por restricciones de Tp,  hasta la 22 inclusive.
\par{Los dos descriptores de segmento de las posiciones 18 y 19 fueron seteados como segmentos de código nivel 0 y 3 respectivamente. Empezando la base de ambos desde la posicion 0x0000, y con un limite de 0x6ffff con el bit de Granularity activado respresentando asi 1.75 GB. Al type de ambos segmentos se les puso el valor de 0xA (Exectute/Read), y los atributos  de Sistema y Precencia en Uno}
\par{Los dos descriptores de segmento de las posiciones 20 y 21 fueron seteados como segmentos de datos nivel 0 y 3 respectivamente. Empezando la base de ambos desde la posicion 0x0000, y con un limite de 0x6ffff con el bit de Granularity activado. Al type de ambos segmentos se les puso el valor de 0x02 (Read/Write), y a los atributos de Sistema y Precencia en Uno.}
\par{Por ultimo al descriptor seteado en la posicion 22 del arreglo, fue colocado como un descriptor de video, con la base a partir de 0xb8000, y limite 0x0f9f, con el bit de Granularity en 0. A este segmento tambien se le asignaron en los atributos de Sistema y Precencia el valor de 1.}


\subsection{Pasar a Modo Protegido}
\par{Luego de cargar la GDT, creandola con los requisitos ya especificados y cargando el GDT register, con lgdt. Habilitamos A20 para poder acceder a las posiciones mayores a 2**20, y seteamos el bit de PE de CR0 en 1.}
\par{Asi ya con todo el contexto armado, ejectuamos la instruccion jmp 18 $<<$ 3:mp para hacer un jmp far a modo protegido donde, ponemos dentro del selector de segmento de codigo, el valor 18 que es la posicion dentro de nuestro arreglo  gdt, donde esta el desciptor de codigo nivel 0.}