\section{Ejercicio 1}
\par{El primer ejercicio consistió en dos elementos, el primero fue sobre llenar la Global Descriptor Table (GDT), con ciertos segmentos y el segundo, en pasar a modo protegido.}

\subsection*{Inicializar la GDT}
\par{En esta parte lo que hicimos primero fue crear un arreglo de gdt\_entry, en el archivo gdt.c de 41 entradas y luego completamos 6 de las mismas. En la primera posición del arreglo fue seteado el descriptor Nulo por restricciones de Intel. Y los otros 5 descriptores de segmento fueron completados empezando desde la posición 18, por restricciones del TP,  hasta la 22 inclusive.
\par{Los dos descriptores de segmento de las posiciones 18 y 19 fueron seteados como segmentos de código nivel 0 y 3 respectivamente. Empezando la base de ambos desde la posición 0x0000, y con un límite de 0x6ffff con el bit de Granularity activado respresentando así 1.75 GB.}
\par{\textcolor{red}{1.75 GB son $\frac{7}{4}*2^{30} B = 7*2^{28} B$. Como hay 20 bits del campo ``Límite'' no iban a ser suficientes para direccionar 1.75 GB como se buscaba, entonces por eso se puso un 1 en el bit de Granularidad para que el Límite se refiriera a bloques de 4 KB. Así, se obtiene $\dfrac{7*2^{28} B}{4 KB} = 7*2^{16}$ y restándole 1 se consigue 0x6FFFF, el contenido que se puso en el campo ``Límite''.}}
\par{Al type de ambos segmentos se les puso el valor de 0xA (Exectute/Read), y los atributos  de Sistema y Presencia en 1.}
\par{Los dos descriptores de segmento de las posiciones 20 y 21 fueron seteados como segmentos de datos nivel 0 y 3 respectivamente. Empezando la base de ambos desde la posición 0x0000, y con un límite de 0x6ffff con el bit de Granularity activado. Al type de ambos segmentos se les puso el valor de 0x02 (Read/Write), y los atributos de Sistema y Presencia en 1.}
\par{Por último el descriptor seteado en la posición 22 del arreglo, fue colocado como un descriptor de video, con la base a partir de 0xb8000, y límite 0x0f9f, con el bit de Granularity en 0. A este segmento también se le asignó en los atributos de Sistema y Presencia el valor 1.}


\subsection*{Pasar a Modo Protegido}
\par{Luego de cargar la GDT, creándola con los requisitos ya especificados y cargando el GDT register con \textbf{lgdt}, habilitamos la línea A20 para poder acceder a las posiciones mayores a $2^{20}$ y seteamos el bit de PE de CR0 en 1.}
\par{Así ya con todo el contexto armado, ejectuamos la instrucción \textbf{jmp} 18 $<<$ 3:mp para hacer un jmp far a modo protegido donde ponemos dentro del selector de segmento de código el valor 18 que es la posición dentro de nuestro arreglo gdt, donde está el desciptor de código nivel 0.}

\clearpage