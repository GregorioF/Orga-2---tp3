\section{Ejercicio 7}
\par{}

\subsection*{Inicialización de las estructuras de datos del \textit{scheduler}}
\par{}


\subsection*{Función sched\_proximo\_indice()}
\par{Primero declaramos dos arreglos de 8 posiciones, uno de tareas y otro de banderas, para indicar si las tareas estaban vivas. Además, usamos un contador current para saber qué tarea correspondía correr.}
\begin{lstlisting}[language={C}]
int tareas[8]={1,1,1,1,1,1,1,1};
int banderas[8]={1,1,1,1,1,1,1,1};
int current = -1;
int currentBanderas = -1;
\end{lstlisting}

\par{Luego implementamos la función \textbf{sched\_proximo\_indice()}. Esta suma 1 a current y entra en un ciclo en el que se aumenta 1 mientras se encuentren tareas que no estén vivas. Si se recorren todas y no hay ninguna para correr entonces current vuelve a -1, en otro caso queda en current mod 8.}
\begin{lstlisting}[language={C}]
short sched_proximo_indice() {
	current +=1 ;
	int i = 0;
	while(tareas[current%8] == 0 && i < 20){
		current +=1;
		i = i+1;
	}
	if(i == 20) return -1;
	return current %8;
}
\end{lstlisting}

\subsection*{Función sched\_proxima\_bandera()}
\par{Usamos un contador currentBanderas para saber qué bandera correspondía correr.}
\begin{lstlisting}[language={C}]
short sched_proxima_bandera(){
	currentBanderas +=1;

	while(tareas[currentBanderas] == 0 ){
		currentBanderas +=1;
		if(currentBanderas == 8) {
			currentBanderas = -1;
			break;
		}
	}
	if( 7 < currentBanderas ) currentBanderas = -1;
	return currentBanderas;
}
\end{lstlisting}

\clearpage