
\section{GDT}
\par{La Tabla de descriptores global o GDT es la encargada de organizar el sistema de segmentación entre otras cosas. Almacena descriptores los cuales pueden ser de segmento, de TSS, de LDT o de call gate. En nuestro sistema contamos con una estructura, gdt$\_$entry, la cual completamos con descriptores de segmentos y de tss.}
\subsection{Descriptores de Segmento}
\par{
Cada descriptor de segmento nos habilita distintas secciones de la memoria. En el caso del tp, nuestro sistema usa segmentación flat, por ende cada descriptor se maneja con el mismo rango de memoria, pero lo que cambia son los atributos con los que se accede a ella.
\begin{itemize}
\item Descriptor Nulo (requerido por Intel).
\item Descriptores Nulos desde la entrada 1 hasta la 17 (requeridos por el tp).
\item Descriptor de Segmento de Código de nivel 0 (privilegios de supervisor y lectura habilitada).
\item Descriptor de Segmento de Código de nivel 3 (privilegios de usuario y lectura habilitada).
\item Descriptor de Segmento de Datos de nivel 0 (privilegios de supervisor y escritura habilitada).
\item Descriptor de Segmento de Datos de nivel 3 (privilegios de usuario y escritura habilitada).
\item Descriptor de Segmento de video 
\end{itemize}
\medskip
\par{Los descriptores previos tenían como base 0x0000, y como límite, 0x6fff, además, el bit de granularidad seteado, por ende cada uno de los segmentos referencia los primeros 1.75 GB. Excepto por el descriptor de segmento de video, el cual posee como base: 0xb800, límite: 0x0f9f y la granularidad en 0.
Todos tienen el bit de sistema en 1 y el de presente también seteado.
}
\medskip

\subsection{Descriptores de TSS}
Luego tenemos las entradas de las tareas, en las cuales profundizaremos más adelante.
\begin{itemize}
\item Descriptor de la Tarea Inicial.
\item Descriptor de la Tarea Idle.
\item De la entrada 25 hasta la 32, nos encontramos con descriptores de tss para cada una de los navíos de nuestro sistema.
\item De la entrada 33 hasta la 40, nos encontramos con descriptores de tss para cada una de las banderas de nuestro tp.
\end{itemize}


\begin{alltt}
\normalfont
		       Todos estos descriptores mantienen el siguiente formato:
		       
         \textbf{Limite:} 0x67
			   
         \textbf{Base:} Dirección de la tss
                     
         \textbf{Presente:} Seteado
                     
         \textbf{Tipo:} 0x9. Combinado con el bit de sistema en 0, tipo 9 se refiere a un descriptor de TSS.
                
         \textbf{Sistema:} 0x0
                     
         \textbf{DPL:} 0x0. Esto es ya que en nuestro sistema queremos que las tareas sólo puedan ser accedidas por
         el kernel, es decir, que no se pueda "saltar" de una tarea a otra.
			 	
         \textbf{Granularidad:} 0x0
                     
         \textbf{AVL:} 0x0
                     
         \textbf{DB:} 0x1 (32 bits)
                     
         \textbf{L:} 0x0
\end{alltt}
}

\section{Modo Protegido}
\par{Al inicio de nuestro sistema nos encontramos en modo real. Esto es consecuencia de la condición de COMPATIBILIDAD de intel. }\par{Al iniciar nuestro sistema tenemos un 8086, nuestro código es de \textbf{16 BITS}; \textbf{NO} hay protección de memoria; podemos utilizar \textbf{TODAS} las instrucciones; AX,CX,DX \textbf{NO} son de propósito general. Y además, sólo podemos direccionar 1 MB de memoria.}\par{ Por esto, luego de cargar los descriptores de código y datos vamos a pasar a "modo protegido", en el cual contamos con código de 32 bits, protección a memoria, y 4 GB de memoria direccionable.
}
\par{
Para poder realizar esto tenemos que encargarnos de un par de puntos previamente:
\begin{itemize}
\item Cargar el GDTR con la dirección de la GDT (LGDTR)
\item Deshabilitar las interrupciones \textbf{EXTERNAS} (CLI)
\item Habilitar A20, es decir, habilitar el acceso a direcciones de memoria superiores a 2$^{20}$.
\item Setear el bit PE del registro CR0. (PE: Protected Mode Enable).
\end{itemize}
Entonces, con el contexto ya armado, ejecutamos la instrucción: \textbf{JMP FAR [selector]:[offset]}. En el caso de nuestro TP, el selector es 18$<<$3, ya que este es el selector del segmento de código de nivel 0; y como offset utilizamos una etiqueta la cual se encuentra inmediatamente a continuación de esta instrucción.
}

\clearpage